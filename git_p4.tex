\documentclass{beamer}
% \setbeameroption{show notes}

\usepackage[utf8x]{inputenc}
\usepackage{ulem} % strike through

\usetheme{Darmstadt}

\title {Every day usage of Git}
\subtitle{with/without Perforce}
\author{Dénes Mátételki}
\institute{www.emerson.com}
\date{October 18, 2013}

\begin{document}

%----------- slide --------------------------------------------------%

\begin{frame}
\titlepage
\end{frame}

%----------- slide --------------------------------------------------%

\begin{frame}
\frametitle{Table of contents}
\tableofcontents
\end{frame}


%----------- slide --------------------------------------------------%

\section{Introduction}

\begin{frame}{The scope}

\begin{block}{About this presentation}
\begin{itemize}
 \item Summarize main advantages of git over other version control systems (SVN, perforce).
 \item Present the most usual, every day process of working with git.
 \item Short summary of how git can work on a perforce system.
 \item No technical depth of git details (non-programmer friendly).
\end{itemize}
\end{block}

\begin{exampleblock}{Please keep it in mind}
The devil is in the (lost) details. It is not the functionality it is the \textit{how} which makes Git an appealing choice. Have an open mindset and give it a try!
\end{exampleblock}


\end{frame}

%----------- slide --------------------------------------------------%

\subsection{History}

\begin{frame}{Short history of Git}

\begin{block}{History}
\begin{itemize}
 \item Initially designed and developed by Linus Torvalds for Linux kernel development in 2005.
 \item The name is a British English slang roughly equivalent to "unpleasant person".
 \item Free and open source. (GNU GPLv2)
\end{itemize}
\end{block}

\begin{block}{Project goals}
\begin{itemize}
 \item Patches to take 3 seconds.
 \item Support a distributed workflow.
 \item Very strong safeguards against corruption.
 \item Performance.
\end{itemize}
\end{block}

\end{frame}

%----------- slide --------------------------------------------------%

\subsection{Comparison}

\begin{frame}{Comparing Git to other VCSs}

\begin{block}{Advantages}
\begin{itemize}
 \item Distributed (offline work, everyone is a full backup).
 \item Very fast (work local files, advanced algorithms).
 \item Small (30x than Subversion. RMS: p4 700M, Git 900M).
 \item Lightweight branches (easy branching, merging).
 \item Flexible, great tools (There is more than one way to do it).
\end{itemize}
\end{block}

\begin{block}{Disadvantages}
\begin{itemize}
 \item Difficult to learn due to complexity.
 \item Revisions don't have version numbers.
 \item \sout{Lacks good GUI tools.}
\end{itemize}
\end{block}

\end{frame}

%----------- slide --------------------------------------------------%

\section{Every day workflow}

\subsection{Collaboration}

\begin{frame}{Every day git - collaboration}

\begin{block}{Workflow}
\begin{itemize}
 \item Update from remote, solve conflicts.
 \item Change local files.
 \item View changes.
 \item Adding them to a changeset.
 \item Create local commit.
 \item Submit local commits to remote.
\end{itemize}
\end{block}

\end{frame}

%----------- slide --------------------------------------------------%

\subsection{Love wolf}

\begin{frame}{Every day git 2 - lone wolf}

\begin{block}{Workflow}
\begin{itemize}
 \item Turn a directory into a Git repository.
 \item Change local files.
 \item View changes.
 \item Adding them to a changeset.
 \item Create local commit.
\end{itemize}
\end{block}

\end{frame}

%----------- slide --------------------------------------------------%

\subsection{With perforce}

\begin{frame}{Git with perforce}

\begin{block}{Git has bridges to most other VCS}
\begin{itemize}
 \item Git repository created from perforce. (separate directory)
 \item Update git from perforce.
 \item Work on git repository (no explicit file checkout).
 \item Submit git commits to perforce.
 \item Commit description can contain defect number and commit can be linked.
\end{itemize}
\end{block}

\begin{exampleblock}{Coexistence}
No need to choose one over the other, both can be used at the same time.
\end{exampleblock}

\end{frame}

%----------- slide --------------------------------------------------%

\section{Usage}

\subsection{History}

\begin{frame}{View history}

\begin{center}
 \includegraphics[height=4cm]{taken.jpg}
\end{center}

\begin{block}{Why? When? By whom?}
\begin{itemize}
 \item View graph of branches.
 \item Search inside of commits.
 \item View changes bentween (ranges of) revisions.
 \item File history.
\end{itemize}
\end{block}

\end{frame}

%----------- slide --------------------------------------------------%

\subsection{Changes}

\begin{frame}{The buckets}

\begin{block}{Git tracks not files but content}
\begin{enumerate}
 \item Untracked files (easily clean them up).
 \item Working tree. (view history).
 \item Changes on local files (list of modified files, diffs).
 \item Staging area for commints. (add diffs, add/remove files).
 \item Local commits (can be edited, reverted).
 \item Remote commits (1984 - Ministry of Truht: don't mess up others' timeline).
\end{enumerate}
\end{block}

\end{frame}

%----------- slide --------------------------------------------------%

\begin{frame}{Handy tools}

\begin{block}{Stash}
\begin{itemize}
 \item Need to switch branches without abandoning or commiting half-done changes.
 \item The changes of local files can be saved as a patches on stash.
 \item The working directory will be clean (before merge, etc)
%  \item Stash content has no parent ( work across branches and commits )
\end{itemize}
\end{block}

\begin{block}{Sending the patch to each other without a central server.}
\begin{itemize}
 \item Local state (diff) can be exported as patch (can be attached to e-mail, or copied from a pendrive).
\end{itemize}
\end{block}

\begin{block}{Taking ownership of patches.}
\begin{itemize}
 \item Patches can applied as commits (by a ``maintainer``).
 \item Cherry pick: apply a commits (from other branches) on current branch.
\end{itemize}
\end{block}

\end{frame}

%----------- slide --------------------------------------------------%

\subsection{Branches}

\begin{frame}{Branches}

\begin{block}{The killer feature}
\begin{itemize}
 \item Switching between branches and commits are very fast (history stored locally, branches are very lightweight)
 \item Quick, effortless local branch creation for each task or feature.
 \item Easy merges and rebases  smart algorithms)
\end{itemize}
\end{block}

\end{frame}

%----------- slide --------------------------------------------------%

\begin{frame}{Advanced}

\begin{block}{Bisect}
\begin{itemize}
 \item Binary search for the commit which introduced a bug.
 \item Marking the last known good state.
 \item Marking a commit known to be broken.
 \item Git selects a commit in the middle, which can be marked again either good or bad.
\end{itemize}
\end{block}

\begin{block}{Subtrees and submodules}
\begin{itemize}
  \item Part of the repositoty is a link to another repository.
  \item Pulling the changes of the ( maybe 3rd party ) subsystem.
  \item Having access to the history of the subsystem.
\end{itemize}
\end{block}

\end{frame}

%----------- slide --------------------------------------------------%

\section{Final thoughts}

\subsection{If I were a sales person}

\begin{frame}{Step right up, step rigth up...}

\begin{block}{To managers - the project should use Git, because:}
\begin{itemize}
 \item It is free and open Source (maintenance)
 \item Most advanced VCS out there, check list of projects using it.
\end{itemize}
\end{block}

\begin{block}{To programmers - use it to track your code, because:}
\begin{itemize}
 \item Fast with advanced tools.
 \item Collaboration and branching and merging are very easy.
 \item Huge community, lots of online help and examples.
\end{itemize}
\end{block}

\begin{block}{To everyone - use it to track your content, because:}
\begin{itemize}
 \item Very easy to use to backup your text files (documents and configuration files).
 \item Lots of convenient services (github, gitourious - free online, firmware of your network attached storage)
\end{itemize}
\end{block}

\end{frame}

%----------- slide --------------------------------------------------%

\begin{frame}{Thank you for your attention!}

\begin{center}

This presentation can be found at:

\small
\url{http://github.com/cs0rbagomba/git_p4/git_p4.pdf}

\smallskip
or
\smallskip

\includegraphics[height=3.3cm]{barcode.png}

\smallskip

Any questions?
\end{center}

\end{frame}

%----------- slide --------------------------------------------------%

\end{document}
